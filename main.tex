\documentclass{EvilStyle}
\usepackage{booktabs}
\usepackage{siunitx}
\usepackage{hyperref}
\usepackage{graphicx}
\usepackage{natbib}
\usepackage{url}

\setlength{\headheight}{21.50314pt}
\newcommand\Ecite[1]{{\setcitestyle{square,super}\cite{#1}}}

\begin{document}

\EvilStylesetup{
  title        = {基于CTPN和CRNN的图像脱敏技术研究和应用},
  title*       = {Research and Application of Image Desensitization Technology \\Based on CTPN and CRNN},
  authors      = {
      {
          name         = {徐佳鼎},
          name*        = {Xujiading},
          affiliations = {上海市计算技术研究所},
          bio          = {男,1997年生,硕士研究生,主要研究领域为机器学习},
          email        = {xujiading1997@163.com},
        },
    },
  affiliations = {
      aff1 = {
          name  = {上海市计算技术研究所,上海市,中国 200040},
          name* = {Shanghai Institute of Computing Technology, Shanghai, China,200040},
        },
    },
  abstract     = {
      随着大数据与人工智能的发展,与之相关各行各业的应用发展也突飞猛进。这其中的基础也是至关重要的就是数据。数据中蕴藏的巨大价值慢慢得到了挖掘并引起了大范围的重视,这也带来了隐私、敏感信息保护方面的棘手难题。本文介绍了图像脱敏的相关方法,并基于CTPN和CRNN提出一种图像脱敏技术,来解决人工脱敏效率低下的问题。并分别对不同样式的两类图片各1000张进行批量脱敏,实验结果表明,该模型可以非常迅速并有效的讲图片中各类的敏感信息全部清洗完毕。
    },
  % 中文关键字与英文关键字对应且一致,应有5-7个关键词
  keywords     = {人工智能, 信息安全,CTPN, CRNN, 图像脱敏},
  abstract*    = {With the development of big data and artificial intelligence, the application development of all walks of life related to it has also made rapid progress. The basis of this is also crucial data. The huge value contained in the data has been gradually mined and attracted a wide range of attention, which also brings the difficult problem of privacy and sensitive information protection. This paper introduces the related methods of image desensitization, and proposes an image desensitization technology based on ctpn and crnn to solve the problem of low efficiency of artificial desensitization. The experimental results show that the model can quickly and effectively clean all kinds of sensitive information in the pictures.},
  % 中文关键字与英文关键字对应且一致,不要用英文缩写
  keywords*    = {artificial intelligence, information safety, CTPN, CRNN,image desensitization},
}

\maketitle


\section{引言}
当今时代,一切都因数据而变革。大数据是当前学术界和产业界的研究热点,正影响着人们日常生活方式、工作习惯及思考模式。所有人都享受着建立在数据之后的应用所带来的的便利,特别是人工智能产品的发展,而其中最应用最广的恐怕就是图像识别的领域。

当下图像识别的方式普遍是采用卷积神经网络\Ecite{CNN}及其后续发展的各类深度学习网络,从大量的图像数据中提取特征,训练模型,从而完成定位、识别、检出的工作。不同图像识别的系统的优劣差异在于,网络模型构建、损失函数设计、学习率的设计以及最重要的数据集的优劣。可以说,如果没有一个优秀的数据集,再完美的算法也无法实现想要的结果\Ecite{karpathy2015deep}。人工智能也就和大数据一起,相辅相成,也同时一起加剧了数据安全问题。因此对于原始数据的保护脱敏工作刻不容缓。

\subsection{背景}

随着我国信息化建设的不断加深,互联网已经深入到了人民日常生活的每个角落,特别是移动互联网的快速发展,大大方便了人们的衣食住行,但是,在享受便利的同时,也给我们带了很多烦恼,比如刚生完孩子就有人打电话推销母婴用品,刚咨询了贷款就有无数的金融平台打电话要提供资金,注册了股票账户就有无数的所谓牛股推荐,在这个数据的时代个人信息会成为很多不法分子的目标之一。敏感数据的泄露,攻击成本低,所带来的收益却极高。

IBM 和Ponemon 联合发布的《2019年度数据泄露成本调研报告》\Ecite{IBMSecurity}对全球16个国家和地区的17个行业的507家公司进行分析。在过去 5 年数据泄露成本上升了 12\%,目前数据泄露的平均成本已经到达了392万美元,这个数字还在不断地增长着。这无论是对个人,企业还是社会来说都是一种无法估计的损失\Ecite{black2013developments}。

相比起其他的数据类型,图像不想文本或者数据库这样是可以结构化的数据。结构化数据的脱敏操作甚至可以采取简单的正则化匹配,就能从数据源过滤掉敏感信息,确保安全。但是图片信息则完全不同,图片无法进行结构化的处理,其中包含敏感信息的方式数不甚数。但值得庆幸的是,图像上敏感信息的泄露主要还是出现在使用的过程中,因此对图像数据的最好的保护就是对原始数据脱敏。

\subsection{传统图像脱敏方法}

% \subsubsection{三级标题}

% \begin{theorem}
%   定理内容。
%   “定义”、“假设”、“公理”、“引理”等的排版格式与此相同,详细定理证明、公式可放在附录中。
% \end{theorem}

% \begin{proof}
%   证明过程.
% \end{proof}

% \begin{figure}[htb]
%   \centering
%   \includegraphics[width=0.7\linewidth]{image/test.pdf}
%   \caption{图片说明 *字体为小 5 号,图片应为黑白图,图中的子图要有子图说明*}
%   \label{test}
% \end{figure}

% \begin{table}[htb]
%   \centering
%   \caption{表说明 *表说明采用黑体*}
%   \small
%   \begin{tabular}{cc}
%     \toprule
%     示例表格 & 第一行为表头,表头要有内容 \\
%     \midrule
%              &                            \\
%     \midrule
%              &                            \\
%     \bottomrule
%   \end{tabular}
% \end{table}

% \begin{procedure}
%   定理内容。
%   “定义”、“假设”、“公理”、“引理”等的排版格式与此相同,详细定理证明、公式可放在附录中。
% \end{procedure}

% \begin{algorithm}
%   定理内容。
%   “定义”、“假设”、“公理”、“引理”等的排版格式与此相同,详细定理证明、公式可放在附录中。
% \end{algorithm}



\section*{致谢}

致谢内容。


\nocite{*}

\bibliographystyle{EvilStyle}
\bibliography{refs}



\appendix

\section{附录}
\end{document}
