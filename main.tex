\documentclass{EvilStyle}

\usepackage{booktabs}
\usepackage{siunitx}
\usepackage{hyperref}
\usepackage{graphicx}

\setlength{\headheight}{21.50314pt}

\begin{document}

\EvilStylesetup{
  title        = {基于CTPN和CRNN的图像脱敏技术研究和应用},
  title*       = {Research and Application of Image Desensitization Technology \\Based on CTPN and CRNN},
  authors      = {
      {
          name         = {徐佳鼎},
          name*        = {Xujiading},
          affiliations = {上海市计算技术研究所},
          bio          = {男,1997年生,硕士研究生,主要研究领域为机器学习},
          email        = {xujiading1997@163.com},
        },
    },
  affiliations = {
      aff1 = {
          name  = {上海市计算技术研究所,上海市,中国 200040},
          name* = {Shanghai Institute of Computing Technology, Shanghai, China,200040},
        },
    },
  abstract     = {
      随着大数据与人工智能的发展,与之相关各行各业的应用发展也突飞猛进。这其中的基础也是至关重要的就是数据。数据中蕴藏的巨大价值慢慢得到了挖掘并引起了大范围的重视,这也带来了隐私、敏感信息保护方面的棘手难题。本文介绍了图像脱敏的相关方法,并基于CTPN和CRNN提出一种图像脱敏技术,来解决人工脱敏效率低下的问题。并分别对不同样式的两类图片各1000张进行批量脱敏,实验结果表明,该模型可以非常迅速并有效的讲图片中各类的敏感信息全部清洗完毕。
    },
  % 中文关键字与英文关键字对应且一致,应有5-7个关键词
  keywords     = {人工智能, 信息安全,CTPN, CRNN, 图像脱敏},
  abstract*    = {With the development of big data and artificial intelligence, the application development of all walks of life related to it has also made rapid progress. The basis of this is also crucial data. The huge value contained in the data has been gradually mined and attracted a wide range of attention, which also brings the difficult problem of privacy and sensitive information protection. This paper introduces the related methods of image desensitization, and proposes an image desensitization technology based on ctpn and crnn to solve the problem of low efficiency of artificial desensitization. The experimental results show that the model can quickly and effectively clean all kinds of sensitive information in the pictures.},
  % 中文关键字与英文关键字对应且一致,不要用英文缩写
  keywords*    = {artificial intelligence, information safety, CTPN, CRNN,image desensitization},
}

\maketitle


\section{一级标题}

\subsection{二级标题}

\subsubsection{三级标题}

\begin{theorem}
  定理内容。
  “定义”、“假设”、“公理”、“引理”等的排版格式与此相同,详细定理证明、公式可放在附录中。
\end{theorem}

\begin{proof}
  证明过程.
\end{proof}

\begin{figure}[htb]
  \centering
  \includegraphics[width=0.7\linewidth]{image/test.pdf}
  \caption{图片说明 *字体为小 5 号,图片应为黑白图,图中的子图要有子图说明*}
  \label{test}
\end{figure}

\begin{table}[htb]
  \centering
  \caption{表说明 *表说明采用黑体*}
  \small
  \begin{tabular}{cc}
    \toprule
    示例表格 & 第一行为表头,表头要有内容 \\
    \midrule
             &                            \\
    \midrule
             &                            \\
    \bottomrule
  \end{tabular}
\end{table}

% \begin{procedure}
%   定理内容。
%   “定义”、“假设”、“公理”、“引理”等的排版格式与此相同,详细定理证明、公式可放在附录中。
% \end{procedure}

% \begin{algorithm}
%   定理内容。
%   “定义”、“假设”、“公理”、“引理”等的排版格式与此相同,详细定理证明、公式可放在附录中。
% \end{algorithm}



\section*{致谢}

致谢内容。


\nocite{*}

\bibliographystyle{EvilStyle}
\bibliography{refs}



\appendix

\section{}

附录内容置于此处,字体为小5号宋体。附录内容包括:详细的定理证明、公式推导、原始数据等



% \begin{background}
% *论文背景介绍为英文,字体为小5号Times New Roman体*

% 论文后面为400单词左右的英文背景介绍。介绍的内容包括:

% 本文研究的问题属于哪一个领域的什么问题。该类问题目前国际上解决到什么程度。

% 本文将问题解决到什么程度。

% 课题所属的项目。

% 项目的意义。

% 本研究群体以往在这个方向上的研究成果。

% 本文的成果是解决大课题中的哪一部分,如果涉及863/973以及其项目、基金、研究计划,注意这些项目的英文名称应书写正确。
% \end{background}

\end{document}
