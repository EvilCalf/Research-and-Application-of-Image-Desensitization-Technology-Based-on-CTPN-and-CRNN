\documentclass{cjc}

\usepackage{booktabs}
\usepackage{siunitx}
\usepackage{hyperref}


\setlength{\headheight}{21.50314pt}


\begin{document}

\cjcsetup{
  title        = {基于CTPN和CRNN的图像脱敏技术研究和应用},
  title*       = {Research and Application of Image Desensitization Technology \\Based on CTPN and CRNN},
  authors      = {
      {
          name         = {徐佳鼎},
          name*        = {Xujiading},
          affiliations = {上海市计算技术研究所},
          bio          = {男,1997年生,硕士研究生,主要研究领域为机器学习},
          email        = {xujiading1997@163.com},
        },
    },
  affiliations = {
      aff1 = {
          name  = {上海市计算技术研究所,上海市,中国 200040},
          name* = {Shanghai Institute of Computing Technology, Shanghai, China,200040},
        },
    },
  abstract     = {
      中文摘要
    },
  % 中文关键字与英文关键字对应且一致,应有5-7个关键词
  keywords     = {关键词, 关键词, 关键词, 关键词},
  abstract*    = {Abstract},
  % 中文关键字与英文关键字对应且一致,不要用英文缩写
  keywords*    = {key word, key word, key word, key word},
}

\maketitle



\section{一级标题}

\subsection{二级标题}

\subsubsection{三级标题}

\begin{theorem}
  定理内容。
  “定义”、“假设”、“公理”、“引理”等的排版格式与此相同,详细定理证明、公式可放在附录中。
\end{theorem}

\begin{proof}
  证明过程.
\end{proof}

\begin{figure}[htb]
  \centering
  \includegraphics[width=\linewidth]{example-fig.pdf}
  \caption{图片说明 *字体为小 5 号,图片应为黑白图,图中的子图要有子图说明*}
\end{figure}

\begin{table}[htb]
  \centering
  \caption{表说明 *表说明采用黑体*}
  \small
  \begin{tabular}{cc}
    \toprule
    示例表格 & 第一行为表头,表头要有内容 \\
    \midrule
             &                            \\
    \midrule
             &                            \\
    \bottomrule
  \end{tabular}
\end{table}

% \begin{procedure}
%   定理内容。
%   “定义”、“假设”、“公理”、“引理”等的排版格式与此相同,详细定理证明、公式可放在附录中。
% \end{procedure}

% \begin{algorithm}
%   定理内容。
%   “定义”、“假设”、“公理”、“引理”等的排版格式与此相同,详细定理证明、公式可放在附录中。
% \end{algorithm}



\section*{致谢}

致谢内容。


\nocite{*}

\bibliographystyle{cjc}
\bibliography{refs}



\appendix

\section{}

附录内容置于此处,字体为小5号宋体。附录内容包括:详细的定理证明、公式推导、原始数据等



% \begin{background}
% *论文背景介绍为英文,字体为小5号Times New Roman体*

% 论文后面为400单词左右的英文背景介绍。介绍的内容包括:

% 本文研究的问题属于哪一个领域的什么问题。该类问题目前国际上解决到什么程度。

% 本文将问题解决到什么程度。

% 课题所属的项目。

% 项目的意义。

% 本研究群体以往在这个方向上的研究成果。

% 本文的成果是解决大课题中的哪一部分,如果涉及863/973以及其项目、基金、研究计划,注意这些项目的英文名称应书写正确。
% \end{background}

\end{document}
